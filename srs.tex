\documentclass{scrreprt}
\usepackage{CJKutf8}
\usepackage{listings}
\usepackage{underscore}
\usepackage[bookmarks=true]{hyperref}
\usepackage[utf8]{inputenc}
\usepackage[english]{babel}
\hypersetup{
    bookmarks=false,    % show bookmarks bar?
    pdftitle={Software Requirement Specification},    % title
    pdfauthor={Jean-Philippe Eisenbarth},                     % author
    pdfsubject={TeX and LaTeX},                        % subject of the document
    pdfkeywords={TeX, LaTeX, graphics, images}, % list of keywords
    colorlinks=true,       % false: boxed links; true: colored links
    linkcolor=blue,       % color of internal links
    citecolor=black,       % color of links to bibliography
    filecolor=black,        % color of file links
    urlcolor=purple,        % color of external links
    linktoc=page            % only page is linked
}%
\def\myversion{1.0 }
\date{}
%\title
\usepackage{hyperref}
\begin{document}
\begin{CJK*}{UTF8}{bsmi}
\begin{flushright}
    \rule{16cm}{5pt}\vskip1cm
    \begin{bfseries}
        \Huge{SOFTWARE REQUIREMENTS\\ SPECIFICATION}\\
        \vspace{1.9cm}
        for\\
        \vspace{1.9cm}
        $<$Number Recognition$>$\\
        \vspace{1.9cm}
        \LARGE{Version \myversion approved}\\
        \vspace{1.9cm}
        Prepared by 1051524 莊子毅\hspace{0.5cm}1052063 朱樂謙\hspace{0.5cm}1053334 陳揆中\hspace{0.5cm}1053342 曹育維\\
        \vspace{1.9cm}
        $<$Team3$>$\\
        \vspace{1.9cm}
        \today\\
    \end{bfseries}
\end{flushright}

\tableofcontents


\chapter*{Revision History}

\begin{center}
    \begin{tabular}{|c|c|c|c|}
        \hline
	    Name & Date & Reason For Changes & Version\\
        \hline
	    21 & 22 & 23 & 24\\
        \hline
	    31 & 32 & 33 & 34\\
        \hline
    \end{tabular}
\end{center}

\chapter{Introduction}

\section{Purpose}
$<$為保持某些重要文件能在公平環境下被辨識的重要系統,以確保每筆資料能透過本系統來公正的判斷數字。因此利用本系統做為一個公正的裁決者,使資料被判讀時不會受人為因素的干擾。.$>$


\section{Intended Audience and Reading Suggestions}
$<\\$本文主要內容共分成下列幾個部分:\\
  1) Product Perspective:如何操作本系統\\
  2) Product Functions:詳細敘述了產品的具體功能\\
  3) Operating Environment:此系統運作平台之規範\\
  4) Functional Requirements:敘述此系統之功能性需求\\
  5) Performance Requirements:敘述此系統之非功能性需求\\
閱讀建議:\\
對於程式開發者: 藉由閱讀文件,能更快了解程式內容。\\
對於程式測試者: 藉由閱讀文件,能更快速解決在測試時遇到的問題(ex.bug)。\\
$>$

\section{Project Scope}
$<$辨識各種類型圖片中的數字,而圖片數字的光暗、角度、形狀、位置的不同,都會影響到程式預測出來的結果。$>$


\chapter{Overall Description}

\section{Product Perspective}
$<$操作流程為將一張內224*224的JPG檔更改為指定名稱後放置在指定資料夾下,並執行程式,程式執行完成後能夠顯示出該圖片與哪一個數字最相似。$>$

\section{Product Functions}
$<\\$
\hspace{2cm}判斷數字\\
\hspace{4cm}-說明: 能夠判定輸入的圖像是甚麼數字\\
\hspace{4cm}-輸入: 224*224的jpg圖,圖片內容為0~9其中一個數字\\
\hspace{4cm}-處理: 判定圖片中的數字與訓練的結果比對最相似於哪個數字\\
\hspace{4cm}-輸出: 顯示與圖片最相似的數字\\$>$

\section{User Classes and Characteristics}
$<$當對於手寫的數字判定與他人分歧時希望有公正判定時,希望能夠透過此軟體解決問題的人為主要客群。$>$

\section{Operating Environment}
$<$本系統適用於Windows作業系統的python3.6環境下。$>$

\section{Design and Implementation Constraints}
$<$限制輸入的圖檔為224*224的JPG圖檔且內容的數字必須是直立的不能太過傾斜。$>$

\section{Assumptions and Dependencies}
$<\\$
  1. 輸入圖檔為224*224的JPG且內容的數字必須是直立的且畫面不能參雜太多額外雜質。\\
  2. 模型訓練能夠依我們期望的訓練成功以符合各式各樣的圖片內容。\\$>$




\chapter{External Interface Requirements}

\section{User Interfaces}
$<$我們的使用者介面是直接使用pycharm來呈現,當您輸入一張圖片,程式會判斷輸入圖片的數字為多少且機率為何。\\例如圖片中的數字為1的機率為0.78、為7的機率為0.22。$>$

\section{Hardware Interfaces}
$<$無。$>$

\section{Software Interfaces}
$<\\$使用Pycharm來編譯程式並執行,其中有安裝到的軟體有:\\
1.	numpy\\
2.	opencv\\
3.	scikit-learn\\
4.	scipy\\
5.	sklearn\\
6.	tensorflow\\
7.	keras\\$>$


\chapter{System Features}


\section{Description and Priority}
$<$由於本系統之目的在於確保數字辨識的正確性,對於數字辨識之準確度的控管相當重要。所以對於資料讀取後的辨識,本系統採取較優先的順序來處理。其他輔助的詳細項目視需求會再陸續增加。$>$

\section{Stimulus/Response Sequences}
$<$當我們將一張224*224(Pixel)的圖片放進對應的資料夾時,按下執行按鈕,等待整個系統執行完成時,系統會將輸出的結果顯示在螢幕上。$>$

\section{Functional Requirements}
$<$程式能夠在1秒鐘內執行出結果,根據輸入的圖片判定出最接近的答案。$>$



\chapter{Other Nonfunctional Requirements}

\section{Performance Requirements}
$<$效能需求描述:\\
  本系統不需要在程式中執行做任何操作,只需要在外部將圖片處理完成,按下程式執行即可跑出結果。$>$

\end{CJK*}
\end{document}
