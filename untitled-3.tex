\documentclass{scrreprt}
\usepackage{CJKutf8}
\usepackage{listings}
\usepackage{underscore}
\usepackage[bookmarks=true]{hyperref}
\usepackage[utf8]{inputenc}
\usepackage[english]{babel}
\hypersetup{
    bookmarks=false,    % show bookmarks bar?
    pdftitle={Software Requirement Specification},    % title
    pdfauthor={Jean-Philippe Eisenbarth},                     % author
    pdfsubject={TeX and LaTeX},                        % subject of the document
    pdfkeywords={TeX, LaTeX, graphics, images}, % list of keywords
    colorlinks=true,       % false: boxed links; true: colored links
    linkcolor=blue,       % color of internal links
    citecolor=black,       % color of links to bibliography
    filecolor=black,        % color of file links
    urlcolor=purple,        % color of external links
    linktoc=page            % only page is linked
}%
\def\myversion{1.0 }
\date{}
%\title
\usepackage{hyperref}
\begin{document}
\begin{CJK*}{UTF8}{bsmi}
\begin{flushright}
    \rule{16cm}{5pt}\vskip1cm
    \begin{bfseries}
        \Huge{SOFTWARE REQUIREMENTS\\ SPECIFICATION}\\
        \vspace{1.9cm}
        for\\
        \vspace{1.9cm}
        $<$Number Recognition$>$\\
        \vspace{1.9cm}
        \LARGE{Version \myversion approved}\\
        \vspace{1.9cm}
        Prepared by 1051524 莊子毅\hspace{0.5cm}1052063 朱樂謙\hspace{0.5cm}1053334 陳揆中\hspace{0.5cm}1053342 曹育維\\
        \vspace{1.9cm}
        $<$Team3$>$\\
        \vspace{1.9cm}
        \today\\
    \end{bfseries}
\end{flushright}

\tableofcontents


\chapter*{Revision History}

\begin{center}
    \begin{tabular}{|c|c|c|c|}
        \hline
	    Name & Date & Reason For Changes & Version\\
        \hline
	    21 & 22 & 23 & 24\\
        \hline
	    31 & 32 & 33 & 34\\
        \hline
    \end{tabular}
\end{center}

\chapter{Introduction}

\section{Purpose}
$<$為保持某些重要文件能在公平環境下被辨識的重要系統,以確保每筆資料能透過本系統來公正的判斷數字。因此利用本系統做為一個公正的裁決者,使資料被判讀時不會受人為因素的干擾。.$>$


\section{Intended Audience and Reading Suggestions}
$<$Describe the different types of reader that the document is intended for, 
such as developers, project managers, marketing staff, users, testers, and 
documentation writers. Describe what the rest of this SRS contains and how it is 
organized. Suggest a sequence for reading the document, beginning with the 
overview sections and proceeding through the sections that are most pertinent to 
each reader type.$>$

\section{Project Scope}
$<$辨識各種類型圖片中的數字,而圖片數字的光暗、角度、形狀、位置的不同,都會影響到程式預測出來的結果。$>$


\chapter{Overall Description}

\section{Product Perspective}
$<$Describe the context and origin of the product being specified in this SRS.  
For example, state whether this product is a follow-on member of a product 
family, a replacement for certain existing systems, or a new, self-contained 
product. If the SRS defines a component of a larger system, relate the 
requirements of the larger system to the functionality of this software and 
identify interfaces between the two. A simple diagram that shows the major 
components of the overall system, subsystem interconnections, and external 
interfaces can be helpful.$>$

\section{Product Functions}
$<\\$
\hspace{2cm}判斷數字\\
\hspace{4cm}-說明: 能夠判定輸入的圖像是甚麼數字\\
\hspace{4cm}-輸入: 224*224的jpg圖,圖片內容為0~9其中一個數字\\
\hspace{4cm}-處理: 判定圖片中的數字與訓練的結果比對最相似於哪個數字\\
\hspace{4cm}-輸出: 顯示與圖片最相似的數字\\$>$

\section{User Classes and Characteristics}
$<$當對於手寫的數字判定與他人分歧時希望有公正判定時,希望能夠透過此軟體解決問題的人為主要客群。$>$

\section{Operating Environment}
$<$本系統適用於Windows作業系統的python3.6環境下。$>$

\section{Design and Implementation Constraints}
$<$此程式設計能夠讀取圖檔,並且判定圖片內容,但限制輸入為224*224的JPG圖檔且內容的數字必須是直立的不能太過傾斜。$>$




\chapter{External Interface Requirements}

\section{User Interfaces}
$<$我們的使用者介面是直接使用pycharm來呈現,當您輸入一張圖片,程式會判斷輸入圖片的數字為多少且機率為何。\\例如圖片中的數字為1的機率為0.78、為7的機率為0.22。$>$

\section{Hardware Interfaces}
$<$Describe the logical and physical characteristics of each interface between 
the software product and the hardware components of the system. This may include 
the supported device types, the nature of the data and control interactions 
between the software and the hardware, and communication protocols to be 
used.$>$

\section{Software Interfaces}
$<\\$使用Pycharm來編譯程式並執行,其中有安裝到的軟體有:\\
1.	numpy\\
2.	opencv\\
3.	scikit-learn\\
4.	scipy\\
5.	sklearn\\
6.	tensorflow\\
7.	keras\\$>$


\chapter{System Features}
$<$This template illustrates organizing the functional requirements for the 
product by system features, the major services provided by the product. You may 
prefer to organize this section by use case, mode of operation, user class, 
object class, functional hierarchy, or combinations of these, whatever makes the 
most logical sense for your product.$>$


\section{Description and Priority}
$<$由於本系統之目的在於確保數字辨識的正確性,對於數字辨識之準確度的控管相當重要。所以對於資料讀取後的辨識,本系統採取較優先的順序來處理。其他輔助的詳細項目視需求會再陸續增加。$>$

\section{Stimulus/Response Sequences}
$<$當我們將一張224*224(Pixel)的圖片放進對應的資料夾時,按下執行按鈕,等待整個系統執行完成時,系統會將輸出的結果顯示在螢幕上。$>$

\section{Functional Requirements}
$<$Itemize the detailed functional requirements associated with this feature.  
These are the software capabilities that must be present in order for the user 
to carry out the services provided by the feature, or to execute the use case.  
Include how the product should respond to anticipated error conditions or 
invalid inputs. Requirements should be concise, complete, unambiguous, 
verifiable, and necessary. Use “TBD” as a placeholder to indicate when necessary 
information is not yet available.$>$

$<$Each requirement should be uniquely identified with a sequence number or a 
meaningful tag of some kind.$>$

REQ-1:	REQ-2:


\chapter{Other Nonfunctional Requirements}

\section{Performance Requirements}
$<$If there are performance requirements for the product under various 
circumstances, state them here and explain their rationale, to help the 
developers understand the intent and make suitable design choices. Specify the 
timing relationships for real time systems. Make such requirements as specific 
as possible. You may need to state performance requirements for individual 
functional requirements or features.$>$

\section{Safety Requirements}
$<$Specify those requirements that are concerned with possible loss, damage, or 
harm that could result from the use of the product. Define any safeguards or 
actions that must be taken, as well as actions that must be prevented. Refer to 
any external policies or regulations that state safety issues that affect the 
product’s design or use. Define any safety certifications that must be 
satisfied.$>$

\section{Security Requirements}
$<$Specify any requirements regarding security or privacy issues surrounding use 
of the product or protection of the data used or created by the product. Define 
any user identity authentication requirements. Refer to any external policies or 
regulations containing security issues that affect the product. Define any 
security or privacy certifications that must be satisfied.$>$


\chapter{Other Requirements}
$<$Define any other requirements not covered elsewhere in the SRS. This might 
include database requirements, internationalization requirements, legal 
requirements, reuse objectives for the project, and so on. Add any new sections 
that are pertinent to the project.$>$

\section{Appendix A: Glossary}
%see https://en.wikibooks.org/wiki/LaTeX/Glossary
$<$Define all the terms necessary to properly interpret the SRS, including 
acronyms and abbreviations. You may wish to build a separate glossary that spans 
multiple projects or the entire organization, and just include terms specific to 
a single project in each SRS.$>$

\section{Appendix B: Analysis Models}
$<$Optionally, include any pertinent analysis models, such as data flow 
diagrams, class diagrams, state-transition diagrams, or entity-relationship 
diagrams.$>$

\section{Appendix C: To Be Determined List}
$<$Collect a numbered list of the TBD (to be determined) references that remain 
in the SRS so they can be tracked to closure.$>$
\end{CJK*}
\end{document}
